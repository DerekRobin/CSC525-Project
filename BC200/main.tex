\documentclass{article}
\usepackage[utf8]{inputenc}
\usepackage{cite}

\begin{document}


\section{What is BC200?}
Brain Cytoplasmic 200 Long Non-coding RNA (BC200) is a 200 nucleotide long RNA transcript which is found mostly in the brain\cite{tiedge1993primary}. As a Non-coding RNA, BC200 is not translated into protein but can be used as a potential therapeutic target and biomarker due to its regulatory role in biological processes involved in disease development\cite{zhang2021role,mus2007dendritic}.This lncRNA has recently been studied extensively because of its role in regulating translation and inhibiting its initiation, as well as its impacts in pathogenesis of Alzheimer's disease and cancer\cite{zhang2021role,tiedge1993primary}.These non-coding RNAs are involved in translation control, thus they impact the synthesis of dendritic proteins which facilitates long-term plastic changes at the synapse.\cite{mus2007dendritic}

\section{What is AD and how BC200 is related to it?}
Alzheimer's disease (AD) is a Neurodegenerative disease resulting from synaptic plasticity failure in neurons\cite{mus2007dendritic}. It is a complex disease, meaning that it involves multiple cell types and signaling pathways\cite{zhang2021role}.
Accumulation of two proteins are the key reasons for AD. One of them is beta-amyloid (A$\beta$) which accumulates in neurons, forms plaques, and disrupts cell functions. The other one is hyperphosphorylated tau protein which in abnormal levels can form neurofibrillary tangles in neurons and block synaptic transmissions. \cite{zhang2021role}.\\
A$\beta$, a cleavage product of the amyloid precursor protein (APP), is generated by b-site APP-cleaving enzyme1 (BACE1) and $\gamma$-secretase complex, and it strongly influences the pathogenesis of AD. Inhibition of BACE1 activity and the subsequent reduction in A$\beta$ levels may cure or prevent AD\cite{li2018identification,zhang2021role}.\\
BC200 facilitates AD pathogenesis by upregulating A$\beta$ production through the modulation of BACE1 expression. The inhibition of BC200 significantly suppresses BACE1 expression, increases cell viability and reduces cell apoptosis in an AD model, and these effects are reversed by BC200 overexpression \cite{li2018identification,zhang2021role}. \\
Many researches have demonstrated the important role of BC200 in AD. El Mus et al\cite{mus2007dendritic} show that there are steady decline in BC200 level from age 49 to 86, But, in AD brain its level was substantially higher. They also observe that BC200 expression is increased in brain areas that are involved in AD and it is parallel with severity of disease. Huanyen Li et al.\cite{li2018identification} establish an AD cell model overexpressing A$\beta$1- 42 to observe the effects of BC200 on the cell viability and apoptosis and to investigate the associated underlying mechanisms. They observe that BC200 and BACE1 were increased upon treatment with A$\beta$1-42, and inhibition of BC200 rescued this A$\beta$1-42-mediated dysfunction, as indicated by the interaction of BC200 directly targeting BACE1. Moreover, inhibition of BC200 increased AD cell growth and reduced cells apoptosis. They demonstrate that BC200 is a potent positive regulator of BACE1 in AD cells and in conclusion, Long noncoding RNA BC200 facilitates AD pathogenesis by upregulating Aβ through BACE1.  					 			 		 	 



- 

\bibliographystyle{IEEEtran}
\bibliography{references.bib}
\end{document}
