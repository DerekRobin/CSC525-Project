\documentclass[conference]{IEEEtran}
\usepackage{cite}
\usepackage{url}
\usepackage{graphicx}

\title{\vspace{-0.5cm}A Research Proposal for \\``Phylogenetic Analysis of lncRNAs Implicated in Alzheimer's Disease''}

\author{Derek Robinson, Mina Emadi, Mazyar Ghezelji\\
University of Victoria\\
Victoria, Canada \\
\{drobinson, minaemadi, mazyarghezelji\}@uvic.ca}

\begin{document}

\maketitle

\section{Motivation}

There is no doubt that Alzheimer's disease (AD) greatly affects the lives of those diagnosed and those who care for the diagnosed. 
In the United States, AD is currently the sixth leading cause of death for American adults \cite{AlzheimersDisease}. 
In Canada, over 747,000 patients are living with AD, or another form of dementia \cite{ADcanada}. 
While many humans suffer from AD, it is not well understood if other great apes (Hominidae) also suffer from AD. 
Finch and Austad argue that with our current understanding of AD, it is not possible to determine if AD is uniquely human \cite{finch2015commentary}.
As a group, we are interested in learning more about AD; specifically, the phylogeny of the long non-coding RNAs (lncRNAs) which have been implicated in AD \cite{luo2016long}. 

\section{Objective}

Our main objective for this research project is to build the phylogenetic tree for one of the lncRNAs implicated in AD, as identified by Luo and Chen \cite{luo2016long}. 
Our current targets are brain cytoplasmic 200 (BC200) and $\beta$-site amyloid precursor protein cleaving enzyme-1 antisense transcript (BACE1-AS). 
Having already performed a preliminary BLAST search identifying several similar sequences to BC200, we will likely choose BC200 as our main target for phylogenetic analysis \cite{madden2012blast,blastTool}. 
Our secondary goal for this research project is to compare the structure of our lncRNA's most closely related homologs to shine light on the possible differences that may lead to AD being uniquely human.

\section{Methods}

The methods we will be applying are those similar to Amirmahani and Goharrizi \cite{amirmahani2018phylogenetic}. 
Specifically, we will be using the bioinformatics program NCBI-BLAST in order to identify homologs of our chosen lncRNA \cite{madden2012blast,blastTool}. 
Once we have identified several homologs of our chosen lncRNA, we will then perform the phylogenetic analysis using MEGA11 \cite{tamura2021mega11}. 
Our phylogenetic tree will be created via all available tree construction algorithms (Maximum Likelihood, Neighbour Joining, and Minimum-Evolution), and each of the results will be compared. 
Once we have built the phylogenetic tree, we will then compare the secondary structure of the three most closely related homologs using RNAz 2.0 \cite{gruber2010rnaz}.

\bibliographystyle{IEEEtran}
\bibliography{refs.bib}
\end{document}